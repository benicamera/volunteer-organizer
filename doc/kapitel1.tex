\chapter{Domain Driven Design}
\label{chapter:domain_driven_design}

\section{Analyse der Ubiquitous Language}
\label{section:analyse_der_ubiquitous_language}

Die Ubiquitous Language dieses Projekts richtet sich nach der üblichen Sprache in der Vereinsdomäne. Die folgenden Begriffe sind speziell aus der Subdomäne \glqq Veranstaltungen\grqq{} entnommen und werden im weiteren Verlauf des Projekts verwendet.

\paragraph{\glqq Veranstaltung\grqq{} (Event)}
Dieser begriff bezeichnet eine Vereinsveranstaltung, die Freiwillige für die Durchführung benötigt. Eine Veranstaltung findet in einem bestimmten Zeitraum statt und hat einen Veranstaltungsort. Außerdem gibt es verschiedene Aufgaben, die in einem bestimmten Zeitfenster durch Freiwillige erledigt werden müssen. Durch die verschiedenen Aufgabenprofile sind auch verschiedene Eigenschaften von Freiwilligen gefragt. Eine Veranstaltung wird durch mindestens einen Organisator organisiert.

\paragraph{\glqq Freiwilliger\grqq{} (Volunteer)}
Dieser Begriff bezeichnet eine Person, die bei einer Veranstaltung des Vereins hilft. Ein Freiwilliger kann verschiedene Eigenschaften besitzen, die für die Durchführung einer Veranstaltung benötigt werden. Ein Freiwilliger kann auch Mitglied im Verein sein.

\paragraph{\glqq Aufgabe\grqq{} (Task)}
Dieser Begriff bezeichnet eine konkrete Tätigkeit, die bei einer Veranstaltung in einem bestimmten Zeitraum durchgeführt werden muss. Außerdem kann es sein, dass eine Aufgabe bestimmte Anforderungen an die Eigenschaften von Freiwilligen hat. Eine Aufgabe wird durch mindestens einen Freiwilligen erledigt.

\paragraph{\glqq Eigenschaft\grqq{} (Feature)}
Dieser Begriff bezeichnet eine Fähigkeit, die ein Freiwilliger besitzt. Neben technischen oder fachlichen Fähigkeiten kann dies auch der Besitz eines Führerscheins oder der Besitz eines Schlüssels für eine bestimmte Räumlichkeit sein.

\paragraph{\glqq Mitglied\grqq{} (Member)}
Dieser Begriff bezeichnet eine Person, die Mitglied im Verein ist. Prinzipiell stehen Mitglieder als potentielle Freiwillige zur Verfügung. Sie haben also auch verschiedene Eigenschaften.

\paragraph{\glqq Veranstaltungsort\grqq{} (Location)}
Dieser Begriff bezeichnet den Ort, an dem eine Veranstaltung stattfindet. Der Ort kann auch \glqq virtuell\grqq{} sein, wenn die Veranstaltung online stattfindet.

\paragraph{\glqq Organisator\grqq{} (Organizer)}
Dieser Begriff bezeichnet ein Vereinsmitglied, das eine Veranstaltung organisiert. Zur Organisation gehört das Festlegen von Ort und Zeit, das Festlegen von Aufgaben und das Einladen von Freiwilligen.

\paragraph{Veranstaltungszeitraum}
Der Veranstaltungszeitraum bezeichnet den Zeitraum, in dem eine Veranstaltung stattfindet.
Dieser Zeitraum kann sich über wenige Stunden, einen Tag oder auch mehrere Tage erstrecken.

\subsection{Domain Events}

\paragraph{\glqq Freiwilliger hat sich für eine Aufgabe gemeldet\grqq}
Dieses Domain Event wird ausgelöst, wenn sich ein Freiwilliger für eine Aufgabe gemeldet hat. Dieses Event kann dazu genutzt werden, um den Organisator zu informieren, dass die Aufgabe besetzt ist.

\paragraph{\glqq Freiwilliger hat sich für eine Aufgabe abgemeldet\grqq}
Dieses Domain Event wird ausgelöst, wenn sich ein Freiwilliger für eine Aufgabe abgemeldet hat. Dieses Event kann dazu genutzt werden, um den Organisator zu informieren, dass die Aufgabe wieder frei ist.

\paragraph{\glqq Veranstaltungsort geändert\grqq}
Dieses Domain Event wird ausgelöst, wenn sich der Veranstaltungsort einer Veranstaltung geändert hat. Dieses Event kann dazu genutzt werden, um die Freiwilligen zu informieren, dass sich der Veranstaltungsort geändert hat.

\paragraph{\glqq Veranstaltungszeitraum geändert\grqq}
Dieses Domain Event wird ausgelöst, wenn sich der Veranstaltungszeitraum einer Veranstaltung geändert hat. Dieses Event kann dazu genutzt werden, um die Freiwilligen zu informieren, dass sich der Veranstaltungszeitraum geändert hat.

\paragraph{\glqq Veranstaltung abgesagt\grqq}
Dieses Domain Event wird ausgelöst, wenn eine Veranstaltung abgesagt wurde. Dieses Event kann dazu genutzt werden, um die Freiwilligen zu informieren, dass die Veranstaltung abgesagt wurde.

\paragraph{\glqq Veranstaltung durchgeführt\grqq}
Dieses Domain Event wird ausgelöst, wenn eine Veranstaltung durchgeführt wurde.

\paragraph{\glqq Veranstaltung erstellt\grqq}
Dieses Domain Event wird ausgelöst, wenn eine Veranstaltung erstellt wurde. Es kann dazu genutzt werden, um die Mitglieder zu informieren, dass eine neue Veranstaltung erstellt wurde.

\section{Analyse und Begründung der verwendeten Muster}
\label{section:analyse_und_begrundung_der_verwendeten_muster}

Ein Veranstaltungsort wird als Value Object modelliert, da er nur durch seine Werte definiert ist (Name und Adresse) und ein Veranstaltungsort immutable ist.

Aus dem gleichen Grund wird auch ein Veranstaltungszeitraum als Value Object modelliert, weil ein Zeitraum unveränderlich durch seine Start- und Endzeit definiert ist.

Dahingegen wird ein Freiwilliger als Entity modelliert, da er eindeutig durch seine Identität (ID-Nummer) definiert ist und sich z.B. seine Eigenschaften, aber auch sein Name ändern kann.
Somit ist ein Freiwilliger mutable. Daraus folgt, dass auch ein Organisator und ein Mitglied als Entity modelliert werden.

\todo{Aggregat vs Entity klären}
Eine Aufgabe ist nach dem Domain Driven Design ein Aggregat, weil es aus mehreren Entities (Freiwillige) und Value Objects (Veranstaltungszeitraum) besteht und eine eigene Lebenszyklus-Logik hat.

Ein Event ist nach Domain Driven Design ein Aggregat, weil es aus mehreren Entities (Aufgaben) und Value Objects (Veranstaltungsort) besteht und eine eigene Lebenszyklus-Logik hat. Ein Event ist also eine Aggregatwurzel, die die Lebenszyklus-Logik für die Entities und Value Objects in einem Event definiert.

